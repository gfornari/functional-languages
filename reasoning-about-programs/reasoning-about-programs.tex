\documentclass[12pt]{article}
 
\usepackage[english]{babel}
\usepackage[margin=1in]{geometry}
\usepackage{verbatim}
\usepackage{amsmath,amsthm,amssymb,amsfonts}

% be aware of the \verb here and there in the code
\newenvironment{code}{\endgraf\verbatim}{\endverbatim}

% source: https://www.overleaf.com/latex/templates/homework-template/bxzrgjkwjpwc
\newenvironment{problem}[2][Problem]{\begin{trivlist}
\item[\hskip \labelsep {\bfseries #1}\hskip \labelsep {\bfseries #2.}]}{\end{trivlist}}

\title{Reasoning about programs}
\author{Giacomo Fornari}
\date{\today}

\begin{document}

\maketitle


\begin{problem}{1}
Using the definition

\begin{code}
[] ++ ys = ys
(x:xs) + ys = x : (xs ++ ys)
\end{code}

verify the following two properties, by induction on \verb|xs|:

\begin{code}
xs ++ [] = xs
xs ++ (ys ++ zs) = (xs ++ ys) ++ zs
\end{code}
\end{problem}


\begin{proof}[Proof of first property]
Base case:

\begin{code}
  [] ++ []
=   { apply ++ }
  []
\end{code}

Inductive case:

\begin{code}
  (x:xs) ++ []
=   { apply ++ }
  x:(xs ++ [])
=   { induction hypothesis }
  x:xs
\end{code}
\end{proof}


\begin{proof}[Proof of second property]
Base case:

\begin{code}
  [] ++ (ys ++ zs)
=   { apply ++ }
  ys ++ zs
=   { unapply inner ++ }
  ([] ++ ys) ++ zs
\end{code}

Inductive case:

\begin{code}
  (x:xs) ++ (ys ++ zs)
=   { apply ++ }
  x:(xs ++ (ys ++ zs))
=   { induction hypothesis }
  x:((xs ++ ys) ++ zs)
=   { unapply ++ }
  (x:(xs ++ ys)) ++ zs
=   { unapply ++ }
  ((x:xs) ++ ys) ++ zs
\end{code}
\end{proof}


\begin{problem}{2}
Show that

\begin{code}
exec (c ++ d) s = exec d (exec c s)
\end{code}

where \verb|exec| is the function that executes the code consisting of sequences of \verb|PUSH n| and \verb|ADD| operations.
\end{problem}

\begin{proof}
Base case:

\begin{code}
  exec ([] ++ d) s
=   { apply ++ }
  exec d s
=   { unapply exec }
  exec d ( exec [] s)
\end{code}

Inductive case of \verb|PUSH n|:

\begin{code}
  exec ((PUSH n:c) ++ d) s
=   { apply ++ }
  exec ((PUSH n):(c ++ d)) s
=   { apply exec }
  exec (c ++ d) (n:s)
=   { induction hypothesis }
  exec d (exec c (n:s))
=   { unapply inner exec }
  exec d (exec (PUSH n:c) s)
\end{code}

Inductive case of \verb|ADD| (we assume that the stack is composed by at least two elements, because otherwise it is not well-formed):

\begin{code}
  exec ((ADD:c) ++ d) (m:n:s)
=   { apply ++ }
  exec (ADD:(c ++ d)) (m:n:s)
=   { apply exec }
  exec (c ++ d) (n+m:s)
=   { induction hypothesis }
  exec d (exec c (n+m:s))
=   { unapply inner exec }
  exec d (exec (ADD:c) (m:n:s))
\end{code}
\end{proof}


\begin{problem}{3}
Given the type and instance declarations below, verify the functor laws for the Tree type, by induction on trees.

\begin{code}
data Tree a = Leaf a | Node (Tree a) (Tree a)

instance Functor Tree where
    --fmap :: (a-> b) -> Tree a -> Tree b
    fmap g (Leaf x) = Leaf (g x)
    fmap g (Node l r) = Node (fmap g l) (fmap g r)
\end{code}
\end{problem}

\begin{proof}[Proof of the first law]
Let's prove that \verb|fmap id = id| holds. We have two different cases: the base case with \verb|Leaf x| and the inductive case with \verb|Node l r|.

Base case:

\begin{code}
  fmap id (Leaf x)
=   { apply fmap }
  Leaf (id x)
=   { apply id }
  Leaf x
=   { unapply id }
  id (Leaf x)
\end{code}

Inductive case:

\begin{code}
  fmap id (Node l r)
=   { apply fmap }
  Node (fmap id l) (fmap id r)
=   { induction hypothesis }
  Node (id l) (id r)
=   { apply id }
  Node l r
=   { unapply id }
  id (Node l r)
\end{code}
\end{proof}

\begin{proof}[Proof of the second law]
For the second law, we have to prove that \verb|fmap (g . f) = fmap g . fmap f|. We have the same two cases as for the proof of the first law.

Base case:

\begin{code}
  fmap (g . f) (Leaf x)
=   { apply fmap }
  Leaf ((g . f) x)
=   { apply . }
  Leaf (g (f x))
=   { unapply fmap }
  fmap g (Leaf (f x))
=   { unapply fmap }
  fmap g (fmap f (Leaf x))
=   { unapply . }
  (fmap g . fmap f) (Leaf x)
\end{code}

Inductive case:

\begin{code}
  fmap (g . f) (Node l r)
=   { apply fmap }
  Node (fmap (g . f) l) (fmap (g . f) r)
=   { induction hypothesis }
  Node ((fmap g . fmap f) l) ((fmap g . fmap f) r)
=   { apply . }
  Node (fmap g (fmap f l)) (fmap g (fmap f r))
=   { unapply outer fmap }
  fmap g (Node (fmap f l) (fmap f r))
=   { unapply inner fmap }
  fmap g (fmap f (Node l r)) 
=   { unapply . }
  (fmap g . fmap f) (Node l r)
\end{code}
\end{proof}


\begin{problem}{4}
Verify the applicative law for the \verb|Maybe| type.
\end{problem}

\begin{proof}[Proof of the first law]
We want to prove that \verb|pure id <*> x = x| holds for \verb|Maybe|.

\verb|Nothing| case:

\begin{code}
  pure id <*> Nothing
=   { apply pure }
  (Just id) <*> Nothing
=   { apply <*> }
  fmap id Nothing
=   { apply fmap }
  Nothing
\end{code}

\verb|Just x| case:

\begin{code}
  pure id <*> (Just x)
=   { apply pure }
  (Just id) <*> (Just x)
=   { apply <*> }
  fmap id (Just x)
=   { apply fmap }
  Just (id x)
=   { apply id }
  Just x
\end{code}
\end{proof}

\begin{proof}[Proof of the second law]
We want to prove that \verb|pure (g x) = pure g <*> pure x| holds for \verb|Maybe|.

\begin{code}
  pure (g x)
=   { apply pure }
  Just (g x)
=   { unapply fmap }
  fmap g (Just x)
=   { unapply <*> }
  (Just g) <*> (Just x)
=   { unapply pure }
  pure g <*> pure x
\end{code}
\end{proof}

\begin{proof}[Proof of the third law]
We want to prove that \verb|x <*> pure y = pure (\g -> g y) <*> x| holds for \verb|Maybe|.

In order to prove the \verb|Nothing| case, we prove that

\begin{code}
x <*> Nothing = Nothing
\end{code}

\begin{proof}[Lemma] \verb|x <*> Nothing = Nothing|

\verb|Nothing| case:

\begin{code}
  Nothing <*> Nothing = Nothing
\end{code}

\verb|Just g| case:

\begin{code}
  (Just g) <*> Nothing
=   { apply <*> }
  fmap g Nothing
=   { apply fmap }
  Nothing
\end{code}
\end{proof}

We can conclude that \verb|Nothing <*> _ = _ <*> Nothing = Nothing| because of the \emph{Lemma} just proved. Therefore, the \verb|Nothing| case for the third law is demonstrated.


\verb|Just x| case:

\begin{code}
  (Just x) <*> pure y
=   { apply pure }
  (Just x) <*> (Just y)
=   { apply <*> }
  fmap x (Just y)
=   { apply fmap }
  Just (x y)
=   { rewrite (x y) using application on a lambda expression }
  Just ((\g -> g y) x)
=   { unapply fmap }
  fmap (\g -> g y) (Just x)
=   { unapply <*> }
  (Just (\g -> g y)) <*> (Just x)
=   { unapply pure }
  pure (\g -> g y) <*> (Just x)
\end{code}
\end{proof}

\begin{proof}[Proof of the fourth law]
We want to prove that \verb|x <*> (y <*> z) = (pure (.) <*> x <*> y) <*> z| holds for \verb|Maybe|.

\verb|Nothing| case:

\begin{code}
  (pure (.) <*> Nothing <*> y) <*> z
=   { apply inner right most <*> }
  (pure (.) <*> Nothing) <*> z
=   { lemma on inner <*> }
  Nothing <*> z
=   { apply <*> }
  Nothing
=   { unapply <*> }
  Nothing <*> (y <*> z)
\end{code}

In the next case we assume that \verb|x|, \verb|y| and \verb|z| are \verb|Just|:

\begin{code}
  (Just x) <*> ((Just y) <*> (Just z))
=   { apply inner <*> }
  (Just x) <*> (fmap y (Just z))
=   { apply <*> }
  fmap x (fmap y (Just z))
=   { unapply . }
  (fmap x . fmap y) (Just z)
=   { unapply the second Functor law }
  fmap (x . y) (Just z)
=   { unapply <*> }
  (Just (x . y)) <*> (Just z)
=   { unapply fmap }
  (fmap (x .) y) <*> (Just z)
=   { unapply <*> }
  ((Just (x .)) <*> y) <*> (Just z)
=   { unapply fmap }
  (fmap (.) (Just x) <*> y) <*> (Just z)
=   { unapply <*> }
  ((Just (.)) <*> (Just x) <*> (Just y)) <*> (Just z)
=   { unapply first pure }
  (pure (.) <*> (Just x) <*> (Just y)) <*> (Just z)
\end{code}
\end{proof}


\begin{problem}{5}
Given the equation \verb|comp' e c = comp e ++ c|, show how to construct the recursive definition for \verb|comp'| by induction on \verb|e|.
\end{problem}

\begin{proof}
The problem is equal to the "vanishing of append". We recall the definition of the \verb|comp| function:

\begin{code}
comp :: Expr -> Code
comp (Val n) = [PUSH n]
comp (Add x y) = comp x ++ comp y ++ [ADD]
\end{code}

Base case:

\begin{code}
  comp' (Val n) c
=   { specification of comp' }
  comp (Val n) ++ c
=   { apply comp }
  [PUSH n] ++ c
=   { apply ++ }
  (PUSH n):c
\end{code}

Inductive case:

\begin{code}
  comp' (Add x y) c
=   { specification of comp' }
  comp (Add x y) ++ c
=   { apply comp }
  comp x ++ comp y ++ [ADD] ++ c
=   { associativity of ++ }
  comp x ++ (comp y ++ [ADD] ++ c)
=   { associativity of ++ }
  comp x ++ (comp y ++ ([ADD] ++ c))
=   { apply inner ++}
  comp x ++ (comp y ++ (ADD:c))
=   { induction hypothesis on inner comp }
  comp x ++ (comp' y (ADD:c))
=   { induction hypothesis }
  comp' x (comp' y (ADD:c))
\end{code}

Concluding, by induction on \verb|e|, we can construct the definition

\begin{code}
comp' :: Expr -> Code -> Code
comp' (Val n) c = (PUSH n):c
comp' (Add x y) c = comp' x (comp' y (ADD:c))
\end{code} 
\end{proof}

\end{document}